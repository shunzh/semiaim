\section{Introduction}
\label{sec:introduction}

Recent robotic car competitions and demonstrations have convincingly
shown that autonomous vehicles are feasible with the current
generation of hardware~\cite{mybib:Darpa07Urban}. Looking ahead to the
time when autonomous cars will be common, Dresner and Stone proposed a
new intersection control protocol called \emph{Autonomous Intersection
Management} (AIM) and showed that by leveraging the control and
network capabilities of autonomous vehicles it is possible to design
an intersection control protocol that is much more efficient than
traffic signals~\cite{bib:Dresner08Multiagent}.  By removing human
factors from control loops, autonomous vehicles, with the help of
advanced sensing devices, can be safer and more reliable than human
drivers.  The AIM protocol exploits the fine control of autonomous
vehicles to allow more vehicles simultaneously to cross an
intersection, thus effectively reducing the delay of vehicles by
orders of magnitude compared to traffic
signals~\cite{mybib:Fajardo12Automated}.

AIM is designed for the time when all vehicles are fully autonomous.  We,
however, anticipate that there will be a long transition period during
which most vehicles have some but not all capabilities of fully
autonomous vehicles.  In fact, this transition period has already
begun. Since the late 1990s, adaptive cruise control systems and lane
departure warning systems have become widely available as optional
equipment on luxury production vehicles. 
% Today's automatic parking
% systems such as those in the Toyota Prius and various BMW models can
% perform parallel parking with little or no human intervention.  While
% AIM provides a significant efficiency improvement at intersections
% when all cars are autonomous, the benefits are minimal even when as
% few as 10\% of the vehicles are driven by humans (Figure 16
% in~\cite{bib:Dresner08Multiagent}).  The requirement that most,
% if not all,
% vehicles are fully autonomous is a key obstacle to the adoption
% of AIM-like intersection control when most vehicles are not fully autonomous.
% 
The National Highway Traffic Safety Administration acknowledges that
fully autonomous vehicles represent just the top level in five levels
of vehicle automation~\cite{bib:NHTSA13Preliminary}. Indeed, they
define a level below this top level with vehicles that
have limited self-driving automation.
The main motivation of this paper is to
propose a new intersection control system called
\emph{Semi-Autonomous Intersection Management (SemiAIM)} that can
accomodate both fully autonomous vehicles and
\emph{semi-autonomous} vehicles with limited self-driving automation.
There is a high likelihood that
human-driven vehicles, semi-autonomous vehicles, and fully autonomous
vehicles will \emph{coexist} on the road in the future.  SemiAIM takes
advantages of this trend and allows autonomous intersections to handle
a traffic mixture with different types of vehicles.

The main contributions of this paper are 1) the introduction of the
concept of SemiAIM; 2) a specification of constraint-based reservation
requests for semi-autonomous vehicles; and 3) detailed empirical
results demonstrating the effectiveness of this protocol, especially
in moderate traffic levels with a mix of human-driven,
semi-autonomous, and fully autonomous vehicles.  The remainder of the
paper is organized as follows.  Section~\ref{sec:aim} outlines the
architecture of AIM which forms the basis of SemiAIM.
Section~\ref{sec:vehicles} and Section~\ref{sec:interface} categorize
semi-autonomous vehicles and discuss the interaction between human
drivers and semi-autonomous vehicles in SemiAIM.
Section~\ref{sec:constraint} describes the constraint-based
reservation system in SemiAIM.  Section~\ref{sec:simulation} presents
the results of the simulation experiments we used to evaluates
SemiAIM.  The related work and the conclusion are given in
Section~\ref{sec:related} and \ref{sec:conclusions}, respectively.


% , a generalization of AIM for semi-autonomous
% vehicles.


% the design of the user interface
% for drivers on semi-autonomous vehicles to interact with SemiAIM.

% In this paper, semi-autonomous vehicles meet the criteria of limited
% self-driving automation (Level 3).
% Another motivation for SemiAim is the drawback of AIM.  

% The main motivation for SemiAim is that we anticipate that there will
% be a long transition period during which most vehicles have some but
% not all capabilities of fully autonomous vehicles.  In fact, this
% transition period has already begun. Since the late 1990s, adaptive
% cruise control systems and lane departure warning systems have become
% widely available as optional equipment on luxury production vehicles.
% Today's automatic parking systems such as those in the Toyota Prius
% and various BMW models can perform parallel parking with little or no
% human intervention. The National Highway Traffic Safety Administration
% defines five levels of vehicle automation~\cite{NHTSA2013}. In this
% paper, semi-autonomous vehicles meet the criteria of limited
% self-driving automation (Level 3). Fully autonomous vehicles are
% equivalent with full self-driving automation (Level 4).  Hence, there
% is an opportunity to develop an intersection control protocol that
% works with these semi-autonomous vehicles.  More importantly, there is
% a high likelihood that human-driven vehicles, semi-autonomous
% vehicles, and fully autonomous vehicles will \emph{coexist} on the
% road in the future.  SemiAIM takes advantages of this trend and allows
% autonomous intersections to handle a traffic mixture with different
% types of vehicles.

% Another motivation for SemiAim is the drawback of AIM.  The AIM policy
% has a significant improvement on the efficiency of intersection, but
% that requires the ratio of fully autonomous vehicles exceed 95\% of
% all type of vehicles. Even if the ratio of fully autonomous vehicles
% exceeds 90\%, there is no significant difference from traffic signal
% policy (Figure 16 in~\cite{bib:Dresner08Multiagent}). 




% In this paper, we introduce a new protocol called
% \emph{Semi-Autonomous Intersection Management} (SemiAIM), which allows
% vehicles with partially-autonomous features such as adaptive cruise
% control to make reservations in AIM.  We demonstrate the feasibility
% for vehicles with limited autonomy to make reservations to enter an
% intersection in AIM-like style.  Our simulation results show that
% traffic delay of semi-autonomous vehicles in SemiAIM is nearly as
% small as the traffic delay of fully-autonomous vehicles in AIM.


% The paper starts with a discussion of the need of a traffic system
% that can handle autonomous, semi-autonomous, and manual-controlled
% vehicles.  Then we discuss the previous work of FCFS-signals and its
% weaknesses.  Then we propose a new AIM-like reservation systems that
% allow human compatible reservation requests.

% Perhaps, with the help of a cellphone, it can send request.
% For example, if the human driver is maintaining a constant speed,
% it will then send the request. If he receive a confirmation
% and believe that the vehicle can continue to maintain the speed,
% it can go through the intersection without stopping.


%%% Local Variables: 
%%% mode: latex
%%% TeX-master: "main"
%%% End:
