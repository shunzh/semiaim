\section{Introduction}

Recent robotic car competitions and demonstrations have convincingly
shown that autonomous vehicles are feasible with the current
generation of hardware~\cite{mybib:Darpa07Urban}. Looking ahead to the
time when autonomous cars will be common, Dresner and Stone proposed a
new intersection control protocol called \emph{Autonomous Intersection
  Management} (AIM) and showed that by leveraging the control and
network capabilities of autonomous vehicles it is possible to design
an intersection control protocol that is much more efficient than
traffic signals~\cite{bib:Dresner08Multiagent}.  By removing human
factors from control loops, autonomous vehicles, with the help of
advanced sensing devices, can be safer and more reliable than human
drivers.  The AIM protocol exploits the fine control of autonomous
vehicles to allow more vehicles simultaneously to cross an
intersection, thus effectively reducing the delay of vehicles by
orders of magnitude compared to traffic
signals~\cite{bib:Fajardo12Automated}.

However, AIM is designed for the time when all, or most, of the
vehicles on the road are autonomous.  Dresner and Stone did propose a
hybrid protocol called FCFS-Signal\footnote{It was originally called
  ``FCFS-Light.''} which allows human-driven vehicles to share the
road with autonomous vehicles at intersections in
AIM~\cite{bib:Dresner07Sharing}. However, their approach has two major
drawbacks.  First, the benefits of FCFS-Signal over conventional
traffic signals are relatively small until 90-95\% of the vehicles on
the road are autonomous.  Second, and most important for this paper,
FCFS-Signal cannot take advantages of the limited autonomous
capabilities of semi-autonomous vehicles.  To remedy these issues,
this paper introduces a new autonomous intersection system called
\emph{SemiAIM} that works not only with autonomous vehicles but also 
with human-driven vehicles and semi-autonomous vehicles.

The main motivation for SemiAim is that we anticipate that there will
be a long transition period during which most vehicles have some but
not all capabilities of fully autonomous vehicles.  In fact, this
transition period has already begun. Since the late 1990s, adaptive
cruise control systems and lane departure warning systems have become
widely available as optional equipment on luxury production vehicles.
Today's automatic parking systems such as those in the Toyota Prius
and various BMW models can perform parallel parking with little or no
human intervention.  Hence, there is an opportunity to develop an
intersection control protocol that works with these semi-autonomous
vehicles.  More importantly, there is a high likelihood that
human-driven vehicles, semi-autonomous vehicles, and fully
autonomous vehicles will \emph{coexist} on the road in the future.
SemiAIM takes advantages of this trend and allows autonomous
intersections to handle a traffic mixture with different types of
vehicles.

The main contributions of this paper are 1) the introduction of the
concept of Semi-Autonomous Intersection Management (SemiAIM); 2) a
full specification of the first SemiAIM multiagent protocol; and 3)
detailed empirical results demonstrating the effectiveness of this
protocol, especially in moderate traffic levels with a mix of
human-driven, semi-autonomous, and fully autonomous vehicles.  The
remainder of the paper is organized as follows.\ldots

% In this paper, we introduce a new protocol called
% \emph{Semi-Autonomous Intersection Management} (SemiAIM), which allows
% vehicles with partially-autonomous features such as adaptive cruise
% control to make reservations in AIM.  We demonstrate the feasibility
% for vehicles with limited autonomy to make reservations to enter an
% intersection in AIM-like style.  Our simulation results show that
% traffic delay of semi-autonomous vehicles in SemiAIM is nearly as
% small as the traffic delay of fully-autonomous vehicles in AIM.


% The paper starts with a discussion of the need of a traffic system
% that can handle autonomous, semi-autonomous, and manual-controlled
% vehicles.  Then we discuss the previous work of FCFS-signals and its
% weaknesses.  Then we propose a new AIM-like reservation systems that
% allow human compatible reservation requests.

% Perhaps, with the help of a cellphone, it can send request.
% For example, if the human driver is maintaining a constant speed,
% it will then send the request. If he receive a confirmation
% and believe that the vehicle can continue to maintain the speed,
% it can go through the intersection without stopping.


%%% Local Variables: 
%%% mode: latex
%%% TeX-master: "main"
%%% End:
