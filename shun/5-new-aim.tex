\section{Constraint-based Reservation Systems}
\label{sec:constraint}

SemiAIM extends AIM by allowing human-driven vehicles and
semi-autonomous vehicles to make reservations in the same way as
autonomous vehicles.  The key idea of SemiAIM is to turn AIM into a
\emph{constraint-based reservation system}, which allows vehicles to
make reservations in terms of constraints over (1) their driving
profile such as their arrival time and arrival velocity, and (2) the
relationships with other vehicles.  In AIM, a reservation request is a
$5$-tuple $\langle l_1, l_2, t_0, v_0, p\rangle$, where $l_1$ is the
entry lane, $l_2$ is the exit lane, $t_0$ is the arrival time, $v_0$
is the arrival velocity, and $p$ is the physical characteristics of
the vehicle.  This information allows the Intersection Manager (IM) to
compute the exact trajectory of the vehicle in the intersection and
reserve tiles for the vehicle on the trajectory.  However, this
computation {assumes} the vehicle can be controlled \emph{precisely} in
the intersection so that it can meet the reservation constraints
exactly.

Human drivers cannot control their vehicles as precisely, and
semi-autonomous vehicles may only be able to control certain aspects
of their trajectories.  However, even partial control can be
sufficient for interfacing with AIM.  For example, vehicles with
cruise control are capable of precisely controlling their speed, even
if a human is steering.  Thus reservations for moving straight through
the intersection may be able to be followed precisely.  Similarly,
vehicles with adaptive cruise control can maintain a certain distance
from the vehicle in front, so they could be able to meet reservations
that are specified relative to the traversal times of other vehicles.
These examples motivate the need for a new reservation system that
relaxes the assumption of exact trajectories so as to allow
semi-autonomous vehicles to make reservations.

To this end, we propose a constraint language to facilitate
communications between vehicles and IMs, which is specified in
Section~\ref{sec:request}.  If a vehicle expresses its reservation
request in this language, the IM will be able to interpret the request
and determine whether it is possible to reserve a matching set of
tiles.  For example, a semi-autonomous vehicle with simple cruise
control can make a reservation stating that it is approaching the
intersection at 30mph and will arrive at the intersection between
10:15:05am and 10:15:10am, and it will go straight through the
intersection.  Upon receiving this reservation request, the IM will
determine the set of tiles along \emph{all} possible trajectories of
the vehicle and check whether any of these tiles have been reserved by
other vehicles.  If none of these tiles is reserved, the IM sends a
confirmation message to the vehicle and the human driver can then turn
on the cruise control accordingly.  %While this %maneuver may seem
dangerous when there are other vehicles in the %intersection, In
practice, the human driver can propose a request with constraints that
are relaxed enough such that he/she can enter the intersection
comfortably and safely, and the IM can guarantee there is no
collision. If the human driver is unable to enter the intersection
according to the proposed reservation, or if the driver does not have
any equipment to make reservations, the human driver must follow the
traffic signals at the intersection.  Thus any possible use of SemiAIM
will be an advantage to the driver.

% \begin{figure}[t]
%   \centering
%   \includegraphics[width=2.45in]{figures/figure1}
%   \caption{Average delay vs. the ratio of autonomous vehicles
% to human-controlled vehicles. Traffic
%     level = 720 veh./hour/lane.}
%   \label{fig:original}
%   \vspace{-.3in}
% \end{figure}

For semi-autonomous vehicles with \emph{adaptive} cruise control, a
reservation request, called an anchor request, proposes that it will
follow the vehicle with a vehicle tag GXC345 in front of it and
maintains a following distance of $5m$.  To determine the set of tiles
for this vehicle, the IM in SemiAIM has to derive the trajectories
from the reservation request of the vehicle GXC345.  Hence, the IM in
SemiAIM will retain all reservation requests in symbolic forms.  If
the vehicle GXC345 is also a semi-autonomous vehicle whose request
depends on other requests, the IM has to deduce the trajectories by
constraint propagation.  Cyclic dependencies may occur if vehicles send
requests simultaneously.\commentp{Won't it be clear which car is
behind the other?  There should never be a proposal to go in *front*
of another car, right?} The IM must break the cycle to prevent
deadlock and let all vehicles enter the intersection eventually.
Anchor reservation requests are described in more detail, along with
constraint-based requests in general, in Section~\ref{sec:request}.


% \begin{figure}[t]
% \centering \fbox{\sf \footnotesize
% \begin{minipage}{3.2in}
% \begin{flushleft}
% \newcommand{\TT}{\hspace*{1em}}
% Procedure \textbf{UpdateEvasionPlanDB}($I$) \\
% \end{flushleft}
% \end{minipage}}
% \caption{This is an algorithm.}
% \label{fig:algm}
% \end{figure}



%%% Local Variables: 
%%% mode: latex
%%% TeX-master: "main"
%%% End:
