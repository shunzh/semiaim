\section{Constraint-based Reservation Systems}
\label{sec:constraint}

SemiAIM extends AIM by allowing human-driven vehicles and
semi-autonomous vehicles to make reservations in the same way as
autonomous vehicles.  The key idea of SemiAIM is to turn AIM into a
\emph{constraint-based reservation system}, which allows vehicles to
make reservations in terms of constraints over (1) their driving
profile such as their arrival time and arrival velocity, and (2) the
relationships with other vehicles.

\subsection{Constraint-Based Requests}

In AIM, a reservation request is a $5$-tuple $\langle l_1, l_2, t_0,
v_0, p\rangle$, where $l_1$ is the entry lane, $l_2$ is the exit lane,
$t_0$ is the arrival time, $v_0$ is the arrival velocity, and $p$ is
the physical characteristics of the vehicle.  This information allows
the Intersection Manager (IM) to compute the exact trajectory of the
vehicle in the intersection and reserve tiles for the vehicle on the
trajectory.  However, this computation assumes the vehicle can be
controlled \emph{precisely} in the intersection so that it can meet
the reservation constraints exactly.  Human drivers cannot control
their vehicles as precisely, and semi-autonomous vehicles may only be
able to control certain aspects of their trajectories.  Therefore, we
need a new kind of reservation requests that do not rely on this
assumption.

We define \emph{constraint-based} reservation requests as follows.
A request message consists of four components:
\begin{enumerate}
\item{\bf Intention}. The direction in which the vehicle intends to
  move.
\item{\bf Vehicle Type}. The type of the vehicle.
\item{\bf Entry Condition}. The condition under which the vehicle
  will enter the intersection.
\item{\bf Acceleration Profile List}. The schedule of acceleration
  the vehicle plans to use to enter an intersection.
\end{enumerate}
The intention of a vehicle is the direction in which the vehicle wants
to exit from an intersection.  The intention is expressed as an
\emph{intention statement}, which is formally a disjunction of lane
and road identifiers: $(l_1 \vee l_2 \ldots \vee r_1 \vee r_2)$, where
$l_i$ is an exit lane and $r_i$ is an exit road. For every lane $l_i$,
there exists only one $r_j$ such that $l_i \in r_j$. 
Examples of legal intentions are $(l_1 \vee l_3)$ or
$(r_1)$.\footnote{An intention in the form of $r_i \vee r_j$ is also possible,
especially for multiple-intersection management, which involves path
planning. In this case, the vehicle is proposing two directions to go
and the IM will respond with a confirmation message for either one. We
will leave this case for future work.}
This feasibility facilitates different path planning strategies the
vehicle might use.

The vehicle type of a vehicle is the information with which the IM can
determine how the vehicle will move inside an intersection.  Different
types of vehicles have different sizes, shapes, and kinematics.  For
example, the motion of a bus is very different from the motion of a
passenger car.  By knowing the type of the vehicle, the IM will be
able to compute the trajectories of the vehicle under various conditions
such as the arrival times and velocities. We assume the IM maintains
a database of vehicle types.

The entry condition is the condition under which a vehicle enters an
intersection.  An \emph{entry statement} is used to describe the entry
condition. An entry statement consists of three parts: the arrival
lane condition, the arrival time constraint and the arrival velocity
constraint.  An \emph{arrival lane condition} states the possible lanes
from which the vehicle will enter the intersection.  It is a disjunction of
labels: $(l_1 \vee l_2 \vee \ldots \vee l_n)$ where $l_i$ is a
possible lane from which the vehicle enters the intersection.  An
\emph{arrival time constraint} $[t_1,\ t_2]$ states the time interval
the vehicle will arrive at the intersection.  An \emph{arrival
velocity constraint} $[v_1,\ v_2]$ states that the arrival velocity of
a vehicle is between $v_1$ and $v_2$.  An entry statement is a
$3$-tuple $\langle (l_1 \vee l_2 \vee \ldots \vee l_n), [t_1,\ t_2],
[v_1,\ v_2] \rangle$.

An \emph{acceleration profile} is the schedule of accelerations the
vehicle will use to accelerate through the intersection on a
trajectory. An acceleration profile is a list of pairs $\langle (t_1,
a_1), (t_2, a_2), \ldots, (t_n, a_n) \rangle \in A$, where $A$ denotes
the set of possible acceleration profiles, $a_i$ is the acceleration
the vehicle will use from time $t_i$ until time $t_{t+1}$.  Note that
the vehicle may or may not provide the acceleration profile of all
possible trajectories in a request message.  If the acceleration
profile is missing, the IM will generate an acceleration profile based
on a simulation of the movement of the vehicle given the vehicle type.
The most trivial acceleration profile is one that gives the vehicle
zero acceleration, indicating that the vehicle will maintain a
constant speed.  Another acceleration profile the IM considers is a
constant acceleration profile.  The IM will consider these
acceleration profiles one by one in order to find one that leads to a
successful reservation.  Ultimately, the reservation includes a
required acceleration profile to be followed by the vehicle once it
enters the intersection.


\subsection{Anchor Requests}
\label{sec:anchor}

Semi-autonomous vehicles with adaptive cruise control use a special
constraint-based reservation request called an \emph{anchor request}.
An anchor request states the intention of following another vehicle.
a vehicle $\veh$ with adaptive cruise control can make an anchor
request such as $\Anchor(\vin, d, T)$, which means that $\veh$ will
start following another vehicle with the VIN number $\vin$ at time
$T$, and then maintain a following distance less than or equal to $d$.
For example, an anchor request $\Anchor(\texttt{GXC345}, {\tt 5m},
\texttt{3:15pm})$ states that the vehicle will follow the vehicle with
a vehicle tag {\tt GXC345} in front of it and maintains a following
distance of $5m$ at \texttt{3:15pm}.

The following distance constraint causes the IM to consider the two
vehicles as being a single, longer vehicle, with no free space between
them.  To determine the set of tiles for this vehicle, the IM in
SemiAIM has to derive the trajectories from the reservation request of
the vehicle GXC345.  Hence, the IM in SemiAIM will retain all
reservation requests in symbolic forms, such that it can compute the
trajectory of the vehicle issuing anchor requests.  Furthermore, if
the vehicle GXC345 in the above example is also a semi-autonomous
vehicle whose request depends on other requests, the IM has to deduce
the trajectories of $\veh$ by constraint propagation.

In their current form, anchor requests are designed specifically for
semi-autonomous vehicles with adaptive cruise control.  This
formulation is general enough to include other kinds of cooperative
maneuver such as platooning~\cite{bib:Sheikholeslam90Longitudinal}.


\subsection{Interpretation of Constraint-based Requests}

Unlike AIM's reservation requests which are precise description of a
particular trajectory, a constraint-based request in SemiAIM is an
\emph{incomplete} description of a trajectory through an intersection.
Therefore, the IM interprets a constraint-based request as a
description of a set of \emph{possible} trajectories that the vehicle
may follow, and reserves all the possible trajectories for the
vehicle.

In many cases, the reservation requests are case-by-case.
However, it follows the same template....


For each, we summarize how the
semi-autonomous vehicle can interact with the IM when authorized to do
so by the driver.  

constant-velocity request

whole-row request


% \begin{small_ind_s_itemize}

% \item \textbf{Adaptive cruise control} 
% A vehicle with this feature
% can propose an \textit{anchor request} that will be discussed in
% Section~\ref{sec:anchor}.  The IM considers whether it can safely
% traverse the intersection by following the vehicle in front of it.

% \item \textbf{Simple Cruise control} A vehicle with this feature can
% propose a \textit{constant-velocity request}. The IM considers whether
% it can traverse the intersection by keeping a constant velocity.

% \item \textbf{Communication device} This is a device, which can be a
% smart-phone or on-board navigation system, that can communicate with
% the Intersection Manager.  It can gather data from the vehicle, and
% communicate instructions to the driver when necessary.  For example,
% at a red signal, the IM could inform the driver that it is now safe to
% enter the intersection.  A vehicle with such a feature could propose a
% \textit{whole-row request} to reserve an entire lane in the
% intersection for the vehicle.  This is a very strong request and is
% only likely to be confirmed in very light traffic in which an entire
% ``row'' is available.

% \end{small_ind_s_itemize}

%%%%%%%%%%%%%%%%%%%%%%%%%%%%%%%%%%%%%%%%%%%%%%%%%%%%%%%%%%%%%%%%%%%


% \noindent \textbf{Type SA-ACC Vehicles} can utilize all of the above
% equipment to make a reservation:
% \begin{enumerate}
%   \setlength{\itemsep}{1pt}
%   \setlength{\parskip}{0pt}
%   \setlength{\parsep}{0pt}

% \item Such a vehicle can propose an anchor request. If the vehicle in
% front of it is autonomous or semi-autonomous and is going in the same
% direction, then if they can both get reservations, the request is
% confirmed. The vehicle can follow the front vehicle and enter the
% intersection.

% \item If the anchor request is denied, it can propose a
% constant-velocity request. If keeping the current velocity it can
% safely traverse the intersection, the request is confirmed.

% \item If the constant-velocity request is denied, it can propose a
% whole-row request. If there is no conflict and the vehicle can enter
% the intersection directly, the request is confirmed.

% \item If denied again, the car must decelerate enough to be able to
% stop before the intersection.  It can retry step 3 or pass control to
% the human.

% \end{enumerate}

% \noindent
% \textbf{Type SA-CC Vehicles} mainly utilize simple cruise control as follows.
% \begin{enumerate}
%   \setlength{\itemsep}{1pt}
%   \setlength{\parskip}{0pt}
%   \setlength{\parsep}{0pt}
% \item Such a vehicle can propose a constant-velocity request. If it
% can enter the intersection by keeping the current velocity, the
% request is confirmed.

% \item If the constant-velocity request is denied, it can propose a whole-row
%   request. If the entire lane is available, the request is confirmed.

% \item If denied again, the car must decelerate enough to be able to
% stop before the intersection.  It can retry step 2 or pass control to
% the human.

% \end{enumerate}

% \noindent
% \textbf{Type SA-Com Vehicles} utilize only communication devices
% to make reservations:
% \begin{enumerate}
%   \setlength{\itemsep}{1pt}
%   \setlength{\parskip}{0pt}
%   \setlength{\parsep}{0pt}
% \item Such a vehicle can propose a whole-row request.  If the entire
% lane is available, the request is confirmed.

% \item If denied again, the car must decelerate enough to be able to
% stop before the intersection.  It can retry step 1 or pass control to
% the human.

% \end{enumerate}


%%%%%%%%%%%%%%%%%%%%%%%%%%%%%%%%%%%%%%%%%%%%%%%%%%%%%%%%%%%%%%%%%%%



% \begin{small_ind_s_itemize}

% \item \textbf{Autonomous vehicles (Type A).} These are fully
% autonomous vehicles that can be totally controlled by computers.

% \item \textbf{Semi-autonomous vehicles (Type SA).}
% Although they are driven by humans, they have some devices that can
% assist human drivers and can communicate with the IM. In this paper,
% we consider three concrete types of SA vehicles, as specified in this
% section.

% \item \textbf{Human-driven vehicles (Type H).}
% These vehicles are exactly the same as the ones on today's roads. They
% are completely controlled by humans and have no communication with the
% IM.

% \end{small_ind_s_itemize}









% In AIM, a reservation request is a $5$-tuple $\langle l_1, l_2, t_0,
% v_0, p\rangle$, where $l_1$ is the entry lane, $l_2$ is the exit lane,
% $t_0$ is the arrival time, $v_0$ is the arrival velocity, and $p$ is
% the physical characteristics of the vehicle.  This information allows
% the Intersection Manager (IM) to compute the exact trajectory of the
% vehicle in the intersection and reserve tiles for the vehicle on the
% trajectory.  However, this computation assumes the vehicle can be
% controlled \emph{precisely} in the intersection so that it can meet
% the reservation constraints exactly.  Human drivers cannot control
% their vehicles as precisely, and semi-autonomous vehicles may only be
% able to control certain aspects of their trajectories.  Therefore, we
% need a new kind of reservation requests that do not rely on this
% assumption.

% \subsection{Examples of Constraint-based Requests and their Interpretations}






%%%%%%%%%%%%%%%%%%%%%%%%%%%%%%%%%%%%%%

% However, even partial control can be sufficient for interfacing with
% AIM.  For example, vehicles with cruise control are capable of
% precisely controlling their speed, even if a human is steering.  Thus
% reservations for moving straight through the intersection may be able
% to be followed precisely.  Similarly, vehicles with adaptive cruise
% control can maintain a certain distance from the vehicle in front, so
% they could be able to meet reservations that are specified relative to
% the traversal times of other vehicles.  These examples motivate the
% need for a new reservation system that relaxes the assumption of exact
% trajectories so as to allow semi-autonomous vehicles to make
% reservations.

% To this end, we propose a constraint language to facilitate
% communications between vehicles and IMs, which is specified in
% Section~\ref{sec:request}.  If a vehicle expresses its reservation
% request in this language, the IM will be able to interpret the request
% and determine whether it is possible to reserve a matching set of
% tiles.  For example, a semi-autonomous vehicle with simple cruise
% control can make a reservation stating that it is approaching the
% intersection at 30mph and will arrive at the intersection between
% 10:15:05am and 10:15:10am, and it will go straight through the
% intersection.  Upon receiving this reservation request, the IM will
% determine the set of tiles along \emph{all} possible trajectories of
% the vehicle and check whether any of these tiles have been reserved by
% other vehicles.  If none of these tiles is reserved, the IM sends a
% confirmation message to the vehicle and the human driver can then turn
% on the cruise control accordingly.
% In practice, the human driver can propose a request with constraints that
% are relaxed enough such that he/she can enter the intersection
% comfortably and safely, and the IM can guarantee there is no
% collision. If the human driver is unable to enter the intersection
% according to the proposed reservation, or if the driver does not have
% any equipment to make reservations, the human driver must follow the
% traffic signals at the intersection.  Thus any possible use of SemiAIM
% will be an advantage to the driver.


%%%%%%%%%%%%%%%%%%%%%%%%%



% Cyclic dependencies may occur if vehicles send
% requests simultaneously.
% \commentp{Won't it be clear which car is
% behind the other?  There should never be a proposal to go in *front*
% of another car, right?} 
% The IM must break the cycle to prevent
% deadlock and let all vehicles enter the intersection eventually.


% \begin{figure}[t]
% \centering \fbox{\sf \footnotesize
% \begin{minipage}{3.2in}
% \begin{flushleft}
% \newcommand{\TT}{\hspace*{1em}}
% Procedure \textbf{UpdateEvasionPlanDB}($I$) \\
% \end{flushleft}
% \end{minipage}}
% \caption{This is an algorithm.}
% \label{fig:algm}
% \end{figure}

% \begin{figure}[t]
%   \centering
%   \includegraphics[width=2.45in]{figures/figure1}
%   \caption{Average delay vs. the ratio of autonomous vehicles
% to human-controlled vehicles. Traffic
%     level = 720 veh./hour/lane.}
%   \label{fig:original}
%   \vspace{-.3in}
% \end{figure}



%%% Local Variables: 
%%% mode: latex
%%% TeX-master: "main"
%%% End:
