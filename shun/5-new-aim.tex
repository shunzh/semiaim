\section{Constraint-based Reservation Systems}
\label{sec:constraint}

SemiAIM extends AIM by allowing human-driven vehicles and
semi-autonomous vehicles to make reservations in the same way as
fully autonomous vehicles.  The key idea of SemiAIM is to turn AIM into a
\emph{constraint-based reservation system}, which allows vehicles to
make reservations in terms of constraints over (1) their driving
profile such as their arrival time and arrival velocity, and (2) the
relationships with other vehicles.

% In this section, we will
% describe this system in detail.
% \subsection{Foundations}
% \subsection{Constraint-Based Requests}

In AIM, a reservation request is a $5$-tuple $\langle l_1, l_2, t_0,
v_0, p\rangle$, where $l_1$ is the entry lane, $l_2$ is the exit lane,
$t_0$ is the arrival time, $v_0$ is the arrival velocity, and $p$ is
the physical characteristics of the vehicle.  This information allows
the Intersection Manager (IM) to compute the exact trajectory of the
vehicle in the intersection and reserve tiles for the vehicle on the
trajectory.  However, this computation assumes the vehicle can be
controlled \emph{precisely} in the intersection so that it can meet
the reservation constraints exactly.  Human drivers cannot control
their vehicles as precisely, and semi-autonomous vehicles may only be
able to control certain aspects of their trajectories.  Therefore, we
need a new kind of reservation requests that do not rely on this
assumption.

\subsection{Maneuverability Profile} 

Semi-autonomous vehicles only have limited self-driving automation.
For example, adaptive cruise control can only provide autonomous
longitudinal control, and still depend on the human driver to use the
steering wheel to maintain the lateral control.  Hence the
longitudinal control of Type SA-ACC vehicles can be very precise, but
the lateral control cannot as precise.  In fact, the human driver
cannot make a sharp turn when the adaptive cruise control is
activated, because human must slow down the vehicle a lot in order to
be able to make a sharp turn precisely.  In our system, we use a
\emph{maneuverability profile} to encode the limitations of the
control of human drivers an autonomy device is activated.
In case of simple cruise control, the maneuverability profile is
written in Lisp's syntax as follows:
{\small
\begin{verbatim}
(cc-profile (v angle)
  (is-auto-speed-control)
  (not is-auto-steering)
  (< velocity v) (> velocity -v)
  (< steer-angle angle) (> steer-angle -angle))
\end{verbatim}
}
where \texttt{v} is the target velocity and and \texttt{angle} is the
maximum feasible steering angle for the human driver when the cruise
control is turned on.  The maneuverability profile for adaptive cruise
control is
{\small
\begin{verbatim}
(acc-profile (vin d derror angle)
  (is-auto-speed-control)
  (not is-auto-steering)
  (< (dist-from vin) (+ d derror))
  (> (dist-from vin) (- d derror))
  (< steer-angle angle) (> steer-angle -angle))
\end{verbatim}
}
where \texttt{vin} is the vehicle number of the vehicle ahead,
\texttt{d} is the target distance from \texttt{vin}, \texttt{derror}
is the maximum error for acc to maintain the target distance from
\texttt{vin}, and \texttt{angle} is the maximum feasible steering
angle for the human driver.

Other types of semi-autonomous vehicle have their own maneuverability
profile.  When a driver agent makes a reservation request, it needs to
make sure that the trajectories encoded by the reservation request
\emph{satisfies} all constraints in the maneuverability profile.

\subsection{Constraint-Based Requests}

Different maneuverability profile can have different sets of
constraints.  To facilitate communication with all kinds of
semi-autonomous vehicles, SemiAIM uses a \emph{unified}
language for vehicles to express their constraints in the same format
in their requests.  We define \emph{constraint-based} reservation
requests as follows.  A request message consists of four components:
\begin{enumerate}
\setlength{\rightmargin}{0em}
\setlength{\leftmargin}{1em}
\setlength{\itemsep}{0em}
\setlength{\topsep}{0em}
\setlength{\parsep}{0em}
\setlength{\parskip}{0pt}
\item{\bf Intention}. The direction in which the vehicle intends to
  move.
\item{\bf Vehicle Type}. The type of the vehicle.
\item{\bf Entry Condition}. The condition under which the vehicle
  will enter the intersection.
\item{\bf Acceleration Profile List}. The schedule of acceleration
  the vehicle plans to use to enter an intersection.
\end{enumerate}
The intention of a vehicle is the direction in which the vehicle wants
to exit from an intersection.  The intention is expressed as an
\emph{intention statement}, which is formally a disjunction of lane
and road identifiers: $(l_1 \vee l_2 \ldots \vee r_1 \vee r_2)$, where
$l_i$ is an exit lane and $r_i$ is an exit road. For every lane $l_i$,
there exists only one $r_j$ such that $l_i \in r_j$. 
Examples of legal intentions are $(l_1 \vee l_3)$ or
$(r_1)$.\footnote{An intention in the form of $r_i \vee r_j$ is also possible,
especially for multiple-intersection management, which involves path
planning. In this case, the vehicle is proposing two directions to go
and the IM will respond with a confirmation message for either one. We
will leave this case for future work.}
This feasibility facilitates different path planning strategies the
vehicle might use.

The vehicle type of a vehicle is the information with which the IM can
determine how the vehicle will move inside an intersection.  Different
types of vehicles have different sizes, shapes, and kinematics.  For
example, the motion of a bus is very different from the motion of a
passenger car.  By knowing the type of the vehicle, the IM will be
able to compute the trajectories of the vehicle under various conditions
such as the arrival times and velocities. We assume the IM maintains
a database of vehicle types.

The entry condition is the condition under which a vehicle enters an
intersection.  An \emph{entry statement} is used to describe the entry
condition. An entry statement consists of three parts: the arrival
lane condition, the arrival time constraint and the arrival velocity
constraint.  An \emph{arrival lane condition} states the possible lanes
from which the vehicle will enter the intersection.  It is a disjunction of
labels: $(l_1 \vee l_2 \vee \ldots \vee l_n)$ where $l_i$ is a
possible lane from which the vehicle enters the intersection.  An
\emph{arrival time constraint} $[t_1,\ t_2]$ states the time interval
the vehicle will arrive at the intersection.  An \emph{arrival
velocity constraint} $[v_1,\ v_2]$ states that the arrival velocity of
a vehicle is between $v_1$ and $v_2$.  An entry statement is a
$3$-tuple $\langle (l_1 \vee l_2 \vee \ldots \vee l_n), [t_1,\ t_2],
[v_1,\ v_2] \rangle$.

An \emph{acceleration profile} is the schedule of accelerations the
vehicle will use to accelerate through the intersection on a
trajectory. An acceleration profile is a list of pairs $\langle (t_1,
a_1), (t_2, a_2), \ldots, (t_n, a_n) \rangle \in A$, where $A$ denotes
the set of possible acceleration profiles, $a_i$ is the acceleration
the vehicle will use from time $t_i$ until time $t_{t+1}$.  Note that
the vehicle may or may not provide the acceleration profile of all
possible trajectories in a request message.  If the acceleration
profile is missing, the IM will generate an acceleration profile based
on a simulation of the movement of the vehicle given the vehicle type
and the entry condition.

% The most trivial acceleration profile is one that gives the vehicle
% zero acceleration, indicating that the vehicle will maintain a
% constant speed.  Another acceleration profile the IM considers is a
% constant acceleration profile.  The IM will consider these
% acceleration profiles one by one in order to find one that leads to a
% successful reservation.  Ultimately, the reservation includes a
% required acceleration profile to be followed by the vehicle once it
% enters the intersection.

Each vehicle has its own algorithm to generate constraint-based
requests that satisfy its maneuverability profile. In this paper, we
do not elaborate how to generate the requests since the algorithms are
different in different vehicles.  However, we define the following two
requests that will be used by Type SA-CC and Type SA-Com vehicles,
respectively:

\begin{small_ind_s_itemize}
\item A \textbf{constant-velocity request}
is $\langle {\sf Intent}, {\sf Type}, {\sf Entry},$ ${\sf AccProfile} \rangle$,
where
${\sf Intent} = \{ l_1 \vee l_2 \vee \ldots \vee l_n \}$
in which $l_i$ is a possible lane from which the vehicle 
enter the intersection;
${\sf Type}$ is the vehicle type;
${\sf Entry} = ({\sf Intent}, [t_1,\ t_2], [v_1,\ v_2])$
is the entry statement; and
${\sf AccProfile} = \langle (t_1, 0) \rangle$
is the acceleration profile.
Since the acceleration is always zero, the vehicle
will move at a constant velocity.

\item A \textbf{whole-row request}:
is $\langle {\sf Intent}, {\sf Type}, {\sf Entry},$ ${\sf AccProfile} \rangle$,
where
${\sf Intent} = \{ l_1 \vee l_2 \vee \ldots \vee l_n \}$
in which $l_i$ is a possible lane from which the vehicle 
enter the intersection;
${\sf Type}$ is the vehicle type;
${\sf Entry} = ({\sf Intent}, [t_1,\ t_2], [v_1,\ v_2])$
is the entry statement; and
${\sf AccProfile}$ is an acceleration profile.
In order to reserve the entire row in an intersection,
the difference between $t_1$ and $t_2$ 
as well as the difference between $v_1$ and $v_2$
must be large enough to cover the entire row.
\end{small_ind_s_itemize}

Unlike AIM's reservation requests which always correspond to a
particular trajectory, a constraint-based request is an
\emph{incomplete} description of the trajectory.  Therefore, the IM
interprets a constraint-based request as a description of a set of
\emph{possible} trajectories that the vehicle may follow, and reserves
all the possible trajectories for the vehicle. Our hypothesis is that
constraint-based requests with more restrictive will perform better,
meaning that some types of semi-autonomous vehicles will work better
with SemiAIM. We will test this hypothesis in
Section~\ref{sec:simulation}.

\subsection{Anchor Requests}
\label{sec:anchor}

Semi-autonomous vehicles with adaptive cruise control use a special
constraint-based request called an \emph{anchor request} to make
reservations. An anchor request is $\langle {\sf Type}, \texttt{vin},
d \rangle$, where ${\sf Type}$ is the vehicle type, \texttt{vin} is
the vehicle number of the vehicle ahead, and $d$ is the following
distance from the vehicle of \texttt{vin}.  
For example, an anchor request $\langle \texttt{truck-with-acc},
\texttt{GXC345}, {\tt 5m} \rangle$ states that the vehicle will
follow the vehicle with a vehicle tag {\tt GXC345} in front of it
and maintains a following distance of $5m$.
This request is special because this request depends on the confirmed
reservation of the vehicle $\veh$ of \texttt{vin}
If $\veh$ does not have a reservation, the IM will reject the anchor
request; otherwise, the IM will copy
the request of $\veh$ and modify the request with the following
distance $d$. 

In their current form, anchor requests are designed specifically for
semi-autonomous vehicles with adaptive cruise control.  This
formulation is general enough to include other kinds of cooperative
maneuver such as platooning~\cite{bib:Sheikholeslam90Longitudinal}.


% Effectively, the IM considers the two
% vehicles as being a single, longer vehicle.
% To determine the set of tiles for this vehicle, the IM in
% SemiAIM has to derive the trajectories from the reservation request of
% the vehicle GXC345.  Hence, the IM in SemiAIM will retain all
% reservation requests in symbolic forms, such that it can compute the
% trajectory of the vehicle issuing anchor requests.  Furthermore, if
% the vehicle GXC345 in the above example is also a semi-autonomous
% vehicle whose request depends on other requests, the IM has to deduce
% the trajectories of $\veh$ by constraint propagation.

% An anchor request states the intention of following another vehicle.
% a vehicle $\veh$ with adaptive cruise control can make an anchor
% request such as $\Anchor(\vin, d, T)$, which means that $\veh$ will
% start following another vehicle with the VIN number $\vin$ at time
% $T$, and then maintain a following distance less than or equal to $d$.




% Cyclic dependencies may occur if vehicles send
% requests simultaneously.
% \commentp{Won't it be clear which car is
% behind the other?  There should never be a proposal to go in *front*
% of another car, right?} 
% The IM must break the cycle to prevent
% deadlock and let all vehicles enter the intersection eventually.




%%%%%%%%%%%%%%%%%%%%%%%%%%%%%%%%%%%%%%

% However, even partial control can be sufficient for interfacing with
% AIM.  For example, vehicles with cruise control are capable of
% precisely controlling their speed, even if a human is steering.  Thus
% reservations for moving straight through the intersection may be able
% to be followed precisely.  Similarly, vehicles with adaptive cruise
% control can maintain a certain distance from the vehicle in front, so
% they could be able to meet reservations that are specified relative to
% the traversal times of other vehicles.  These examples motivate the
% need for a new reservation system that relaxes the assumption of exact
% trajectories so as to allow semi-autonomous vehicles to make
% reservations.

% To this end, we propose a constraint language to facilitate
% communications between vehicles and IMs, which is specified in
% Section~\ref{sec:request}.  If a vehicle expresses its reservation
% request in this language, the IM will be able to interpret the request
% and determine whether it is possible to reserve a matching set of
% tiles.  For example, a semi-autonomous vehicle with simple cruise
% control can make a reservation stating that it is approaching the
% intersection at 30mph and will arrive at the intersection between
% 10:15:05am and 10:15:10am, and it will go straight through the
% intersection.  Upon receiving this reservation request, the IM will
% determine the set of tiles along \emph{all} possible trajectories of
% the vehicle and check whether any of these tiles have been reserved by
% other vehicles.  If none of these tiles is reserved, the IM sends a
% confirmation message to the vehicle and the human driver can then turn
% on the cruise control accordingly.
% In practice, the human driver can propose a request with constraints that
% are relaxed enough such that he/she can enter the intersection
% comfortably and safely, and the IM can guarantee there is no
% collision. If the human driver is unable to enter the intersection
% according to the proposed reservation, or if the driver does not have
% any equipment to make reservations, the human driver must follow the
% traffic signals at the intersection.  Thus any possible use of SemiAIM
% will be an advantage to the driver.


% \begin{figure}[t]
% \centering \fbox{\sf \footnotesize
% \begin{minipage}{3.2in}
% \begin{flushleft}
% \newcommand{\TT}{\hspace*{1em}}
% Procedure \textbf{UpdateEvasionPlanDB}($I$) \\
% \end{flushleft}
% \end{minipage}}
% \caption{This is an algorithm.}
% \label{fig:algm}
% \end{figure}



%%% Local Variables: 
%%% mode: latex
%%% TeX-master: "main"
%%% End:
