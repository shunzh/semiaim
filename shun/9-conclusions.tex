\section{Conclusions and Future Work}
\label{sec:conclusions}

This paper introduces SemiAIM, a new constraint-based autonomous
intersection management system that enables human-driven vehicles and
semi-autonomous vehicles, in addition to fully autonomous vehicles, to
make reservations and enter an intersection in an AIM-like style. To
the best of our knowledge, SemiAIM is the first intersection control
protocol to leverage limited autonomy of robotic cars to enable smooth
interactions between human-driven, fully autonomous, and
semi-autonomous vehicles at intersections. Our experimental results
showed that our system can greatly decrease traffic delay when most
vehicles are semi-autonomous, even when few (if any) are fully
autonomous. Our incremental deployment study shows that traffic delay
keeps decreasing as more vehicles employ features of autonomy.  In the
future, we intend to devise better constraint-based reservation
requests using more accurate profiling of the vehicles' physical
behavior.


% This study opens up several interesting directions for future work.
% For example, an open question is how to design better constraint-based
% reservation requests using more accurate profiling of the vehicles'
% physical behavior.  It will also be important to study in detail the
% performance of SemiAIM under a variety of different, or varying,
% traffic levels, and with different amounts of traffic traveling in
% different directions.



% the different types of
% constraint-based reservation requests affect the efficiency of
% SemiAIM.  Some reservation requests are much less efficient than the
% others, and how to design 
% efficient reservation requests is an
% important topic for the future.  




% ---not restricted to ones
% described in this paper, but general reservation requests that can be
% understood by the intersection manager---


% \commentp{Isn't that what our experiments already do?  What's
% different here?} \commentn{I think this is what Chiu means..} 


%%% Local Variables: 
%%% mode: latex
%%% TeX-master: "main"
%%% End:
