\section{Semi-Autonomous Vehicles}
\label{sec:vehicles}

Dresner and Stone proposed the FCFS-signal policy, which makes AIM
compatible with human drivers~\cite{bib:Dresner07Sharing}.  According
to their study, when more than 5-10\% of traffic is comprised of
human-driven vehicles, the average delay time of all vehicles
increases significantly.  For one thing, the human-driven vehicles
themselves may not enter the intersection during red signal phases.
But more importantly, they prevent the autonomous vehicles from
obtaining reservations when they are behind such stalled vehicles.  To
allow more efficient management of traffic at intersections, we
consider human-driven vehicles with additional communication features
to facilitate interaction with the IM and other vehicles. In this
paper, we use the term \emph{semi-autonomous vehicles} to refer to
both the vehicles with simple/adaptive cruise control and human-driven
vehicles with additional communication capabilities. We consider the
following categories of vehicles.

\begin{itemize}

\item \textbf{Autonomous vehicles (Type A).} These are fully
autonomous vehicles that can be totally controlled by computers.

\item \textbf{Semi-autonomous vehicles (Type SA).}
Although they are driven by humans,
they have some devices that can assist human drivers and
can communicate with the IM. In this paper, 
we consider three concrete types of SA vehicles, 
as specified in this section.

\item \textbf{Human-driven vehicles (Type H).}
These vehicles are exactly the same as the ones on today's
roads. They are completely controlled by humans and have no
communication with the IM.

\end{itemize}

Moreover, we classify semi-autonomous vehicles according to the
equipment on the vehicles. We consider the following set of
equipment that can be put on a semi-autonomous vehicle.  This
equipment is based on technology that is readily available today.

\begin{itemize}

\item \textbf{Adaptive cruise control} A vehicle can be set to follow
another vehicle in front of it.  This feature is currently available
as an option in some high-end vehicles. A vehicle with this feature
can propose an \textit{anchor request} that will be discussed in
Section~\ref{sec:request}.  The IM considers whether it can safely
traverse the intersection by following the vehicle in front of it.

\item \textbf{Simple Cruise control} A vehicle can be set to maintain
a constant velocity by turning on cruise control. This is a technology
that is widely available today. A vehicle with this feature can
propose a \textit{constant-velocity request}. The IM considers whether
it can traverse the intersection by keeping a constant velocity.

\item \textbf{Communication device} This is a device, which can be a
smart-phone or on-board navigation system, that can communicate with
the Intersection Manager.  It can gather data from the vehicle, and
communicate instructions to the driver when necessary.  For example,
at a red signal, the IM could inform the driver that it is now safe to
enter the intersection.  A vehicle with such a feature could propose a
\textit{whole-row request} to reserve an entire lane in the
intersection for the vehicle.  This is a very strong request and is
only likely to be confirmed in very light traffic in which an entire
``row'' is available.

% \commentp{As described in Section III? (see next comment), this
%   request is a...}

% \commentp{This is a significant weakness of the paper.  "Whole-row
% request", "constant-velocity request", and "anchor request" haven't
% been defined yet.  They're used below, but really only "anchor
% request" is defined later in the paper (section VI). I think it makes
% more sense to merge section VI into section III and clearly define the
% types of requests that can be defined using constraints.  These
% requests could also be used by fully autonomous vehicles, so there's
% no reason that it needs to happen after section IV.  We can then
% emphasize that the traditional AIM system can also be implemented
% using these constraint-based requests (since a normal "fixed arrival,
% fixed trajectory" request is also possible).}

\end{itemize}

All of this equipment gives semi-autonomous vehicles \emph{some} of
the functionality of autonomous vehicles, even though human drivers
still retain some control of the vehicles.  Next, we introduce several
types of semi-autonomous vehicles that we envision utilizing this
equipment.  For each, we summarize how the semi-autonomous vehicle can
interact with the IM when authorized to do so by the driver.  In
general, having additional equipment provides additional potential
advantages over manually driven cars.  But we emphasize that in all
cases, the human driver can always opt to simply follow the traffic
signals if engaging the equipment is too distracting at any given
time.  Table~\ref{table:type} summarizes the equipment being used on
different types of the semi-autonomous vehicles that we consider.

\begin{table}
\centering
\caption{Features of semi-autonomous vehicles.}
\label{table:type}
\begin{tabular}{|c|c|c|c|}
  \hline
  Vehicle Type & Communication & Cruise & Adaptive \\
               & device & control & cruise control \\
  \hline
  SA-ACC & X & X & X  \\
  \hline
  SA-CC & X & X &  \\
  \hline
  SA-Com & X & &  \\
  \hline
\end{tabular}
\end{table}

\noindent \textbf{Type SA-ACC Vehicles} can utilize all of the above
equipment to make a reservation:
\begin{enumerate}

\item Such a vehicle can propose an anchor request. If the vehicle in
front of it is (semi-)autonomous and is going in the same direction,
then if they can both get reservations, the request is confirmed. The
vehicle can follow the front vehicle and enter the intersection.

\item If the anchor request is denied, it can propose a
constant-velocity request. If keeping the current velocity it can
safely traverse the intersection, the request is confirmed.

\item If the constant-velocity request is denied, it can propose a
whole-row request. If there is no conflict and the vehicle can enter
the intersection directly, the request is confirmed.

\item If denied again, the car must decelerate enough to be able to
stop before the intersection.  It can retry step 3 or pass control to
the human.

\end{enumerate}

\noindent
\textbf{Type SA-CC Vehicles} mainly utilize simple cruise control as follows.
\begin{enumerate}

\item Such a vehicle can propose a constant-velocity request. If it
can enter the intersection by keeping the current velocity, the
request is confirmed.

\item If the constant-velocity request is denied, it can propose a whole-row
  request. If the entire lane is available, the request is confirmed.

\item If denied again, the car must decelerate enough to be able to
stop before the intersection.  It can retry step 2 or pass control to
the human.

\end{enumerate}

\noindent
\textbf{Type SA-Com Vehicles} utilize only communication devices
to make reservations:
\begin{enumerate}

\item Such a vehicle can propose a whole-row request.  If the entire
lane is available, the request is confirmed.

\item If denied again, the car must decelerate enough to be able to
stop before the intersection.  It can retry step 1 or pass control to
the human.

\end{enumerate}

In Figure~\ref{fig:demo}, we color Type A vehicles yellow, Type SA-ACC vehicles
green, Type SA-CC vehicles blue, Type SA-Com vehicles white and Type H
vehicles magenta. The figure shows a red phase for all lanes, Some Type A vehicles
are permitted to enter the intersection.



% Our vehicle types
% SA-ACC, SA-CC, and SA-Com have combinations of these features as
% summarized in 

% Adaptive cruise control
% Used by Type SA-ACC vehicles,

% \item \textbf{Simple Cruise control} (Used by Types SA-ACC and SA-CC
%   vehicles).

% \item \textbf{Communication device} (Used Types SA-ACC, SA-CC, and SA-Com vehicles).



%%% Local Variables: 
%%% mode: latex
%%% TeX-master: "main"
%%% End:
