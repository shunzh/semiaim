\section{Semi-Autonomous Vehicles}
\label{sec:vehicles}

A drawback of FCFS is that it does not work with any vehicles that
require human operations.  Dresner and Stone proposed FCFS-signal
(a.k.a. FCFS-Light) to make AIM compatible with human
drivers~\cite{bib:Dresner07Sharing}.  According to their study,
however, when more than 5-10\% of traffic is comprised of human-driven
vehicles, the average delay time of all vehicles increases
significantly under FCFS-signal.  The reason for this phenomenon is
twofold.  First, human-driven vehicles cannot enter the intersection
during red signal phases, because they are \emph{not} allowed to make
any reservation.  Second, as a consequence of the first, autonomous
vehicles often fail to obtain reservations when they are behind the
human-driven vehicles stalled at red signals. These issues stem from
the fact that FCFS-signal prohibits human-driven vehicles from entering
intersections during red signal phases. SemiAIM sets out to overcome
these issues by allowing \emph{semi-autonomous vehicles}---vehicles
with limited autonomous driving and wireless communication
capabilities---to use the AIM reservation system to enter an
intersection during red signal phases.  While these vehicles are not
fully autonomous, they are able to follow a \emph{limited} number of
predictable trajectories at intersections more precisely than human
drivers.  This ability allows them to utilize our constraint-based
reservation system to make reservations in the same manner as fully
autonomous vehicles and enter an intersection at red signals.

Our proposed reservation system is general enough to accept
reservation requests from \emph{any} semi-autonomous vehicles that are
capable of following some trajectories and communicating with the IM.
To facilitate our discussion, we will focus on semi-autonomous
vehicles which use the following set of equipment that is readily
available today.

\begin{small_ind_s_itemize}

\item \textbf{Communication device (Com)}:
a component in a vehicle's on-board electronic system that enables the
vehicle to wirelessly communicate with the transportation
infrastructure including the IM. The communication is bidirectional.
% the messages sent from the IM is presented to the human driver on
% the LCD screen of an on-board navigation system or on a smartphone, and
% the human driver makes decisions on the user interface of the device.
% The device is also hooked up with the odometer, GPS, and other sensors
% such that it can send these sensing information along with the request
% messages to the IM.

\item \textbf{Simple Cruise control (CC)}:
An optional speed control subsystem in vehicles' drivetrain that
automatically controls the vehicle speed by taking over the throttle
of the vehicles.
% With the help of cruise control systems, vehicles
% can maintain a steady constant velocity more precisely than human
% drivers can manually.

\item \textbf{Adaptive cruise control (ACC)}:
an advanced cruise control system that automatically adjusts the speed
of a vehicle in order to maintain a certain distance from vehicles
ahead.
% To achieve this car-following maneuver,
% ACC uses on-board distance sensors coupled with cruise control
% in a feedback loop.

% This feature is currently available
% as an option in some high-end vehicles.

% A vehicle with this feature
% can propose an \textit{anchor request} that will be discussed in
% Section~\ref{sec:anchor}.  The IM considers whether it can safely
% traverse the intersection by following the vehicle in front of it.


% This is a technology that is widely available today. 

% A vehicle with this feature can
% propose a \textit{constant-velocity request}. The IM considers whether
% it can traverse the intersection by keeping a constant velocity.


% This is a device, which can be a
% smart-phone or on-board navigation system, that can communicate with
% the Intersection Manager.  It can gather data from the vehicle, and
% communicate instructions to the driver when necessary.  For example,
% at red signals the IM can inform the driver that it is now safe to
% enter the intersection via the communication device.

% A vehicle with such a feature could propose a
% \textit{whole-row request} to reserve an entire lane in the
% intersection for the vehicle.  This is a very strong request and is
% only likely to be confirmed in very light traffic in which an entire
% ``row'' is available.

\end{small_ind_s_itemize}

All of this equipment give semi-autonomous vehicles \emph{some} of
the functionality of autonomous vehicles, though human drivers still
retain some control of the vehicles.  We can equip a semi-autonomous
vehicle with more than one of these devices. In this paper we introduce three
types of semi-autonomous vehicles that we envision utilizing this
equipment: \textit{Type SA-ACC Vehicles}, \textit{Type SA-CC Vehicles}
and \textit{Type SA-Com Vehicles}.  Table~\ref{table:type} summarizes
the equipment used on these different types of semi-autonomous
vehicles.  Figure~\ref{fig:simulator} shows a screenshot of the
SemiAIM simulator we developed to simulate semi-autonomous vehicles at
intersections. In this figure, the yellow vehicles are fully
autonomous, the green vehicles are Type SA-ACC, the blue vehicles are
Type SA-CC, the white vehicles are Type SA-Com, and the magenta
vehicles are human-driven vehicles.


% \begin{small_ind_s_itemize}
% \item \textbf{Type SA-ACC Vehicles}: Utilizing adaptive cruise
% control to enter an intersection by either moving straight through the
% intersection or following another vehicle.
% \item \textbf{Type SA-CC Vehicles}: Using simple cruise control only
% to enter an intersection at a constant velocity in a straight line.
% \item \textbf{Type SA-Com Vehicles}: Reserve an entire lane in an
% intersection such that the human driver can get through the
% intersection without the help of any autonomous control device; thus
% only the communication device is needed.
% \end{small_ind_s_itemize}

\begin{table}
\centering
\caption{Features of semi-autonomous vehicles.}
\label{table:type}
\small
\vspace{-.1in}
\begin{tabular}{|c|c|c|c|}
  \hline
  Vehicle Type & Communication & Cruise & Adaptive \\
               & Device & Control & Cruise Control \\
  \hline
  SA-ACC & X & X & X  \\
  SA-CC & X & X &  \\
  SA-Com & X & &  \\
  \hline
\end{tabular}
\vspace{-.3in}
\end{table}

% As can be seen, a
% vehicle can utilize several devices at the same time.  In general,
% having additional equipment provides additional potential advantages
% over manually driven cars.  Since adaptive cruise control can work as
% simple cruise control, Type SA-ACC vehicles subsume Type SA-CC
% vehicles, which in turn subsume Type SA-Com vehicles.  Of course, all
% these vehicles can be manual-driven, because the human driver can
% always opt to simply follow the traffic signals without using cruise
% control, especially when engaging the equipment is too distracting.
% Figure~\ref{fig:simulator} shows a screenshot of the SemiAIM simulator
% we developed to simulate semi-autonomous vehicles at
% intersections.
% \footnote{Modified from the open-source AIM4 simulator:
% \url{http://www.cs.utexas.edu/~aim}}.
% In this figure, the yellow
% vehicles are fully autonomous, the green vehicles are Type SA-ACC, the
% blue vehicles are Type SA-CC, the white vehicles are Type SA-Com, and
% the magenta vehicles are human-driven vehicles.
% Although the traffic
% signals at the intersection are red (see the red dots at the edges of
% the intersection), fully autonomous vehicles can still enter the
% intersection. Actually, all vehicles except the human-driven vehicles
% are eligible to enter the intersection as long as they get
% reservations from the IM.





% \noindent \textbf{Type SA-ACC Vehicles} can utilize all of the above
% equipment to make a reservation:
% \begin{enumerate}
%   \setlength{\itemsep}{1pt}
%   \setlength{\parskip}{0pt}
%   \setlength{\parsep}{0pt}

% \item Such a vehicle can propose an anchor request. If the vehicle in
% front of it is autonomous or semi-autonomous and is going in the same
% direction, then if they can both get reservations, the request is
% confirmed. The vehicle can follow the front vehicle and enter the
% intersection.

% \item If the anchor request is denied, it can propose a
% constant-velocity request. If keeping the current velocity it can
% safely traverse the intersection, the request is confirmed.

% \item If the constant-velocity request is denied, it can propose a
% whole-row request. If there is no conflict and the vehicle can enter
% the intersection directly, the request is confirmed.

% \item If denied again, the car must decelerate enough to be able to
% stop before the intersection.  It can retry step 3 or pass control to
% the human.

% \end{enumerate}

% \noindent
% \textbf{Type SA-CC Vehicles} mainly utilize simple cruise control as follows.
% \begin{enumerate}
%   \setlength{\itemsep}{1pt}
%   \setlength{\parskip}{0pt}
%   \setlength{\parsep}{0pt}
% \item Such a vehicle can propose a constant-velocity request. If it
% can enter the intersection by keeping the current velocity, the
% request is confirmed.

% \item If the constant-velocity request is denied, it can propose a whole-row
%   request. If the entire lane is available, the request is confirmed.

% \item If denied again, the car must decelerate enough to be able to
% stop before the intersection.  It can retry step 2 or pass control to
% the human.

% \end{enumerate}

% \noindent
% \textbf{Type SA-Com Vehicles} utilize only communication devices
% to make reservations:
% \begin{enumerate}
%   \setlength{\itemsep}{1pt}
%   \setlength{\parskip}{0pt}
%   \setlength{\parsep}{0pt}
% \item Such a vehicle can propose a whole-row request.  If the entire
% lane is available, the request is confirmed.

% \item If denied again, the car must decelerate enough to be able to
% stop before the intersection.  It can retry step 1 or pass control to
% the human.

% \end{enumerate}


% \begin{small_ind_s_itemize}

% \item \textbf{Autonomous vehicles (Type A).} These are fully
% autonomous vehicles that can be totally controlled by computers.

% \item \textbf{Semi-autonomous vehicles (Type SA).}
% Although they are driven by humans, they have some devices that can
% assist human drivers and can communicate with the IM. In this paper,
% we consider three concrete types of SA vehicles, as specified in this
% section.

% \item \textbf{Human-driven vehicles (Type H).}
% These vehicles are exactly the same as the ones on today's roads. They
% are completely controlled by humans and have no communication with the
% IM.

% \end{small_ind_s_itemize}




% simple/adaptive cruise control and/or
% human-driven vehicles with additional communication capabilities.

% To allow more efficient management of traffic at intersections, we
% consider human-driven vehicles with additional communication features
% to facilitate interaction with the IM and other vehicles.

% While our proposed reservation system is general enough to handle many
% different kinds of semi-autonomous vehicles, our discussion will focus
% on the following categories of vehicles.

% \item \textbf{Simple Cruise control} (Used by Types SA-ACC and SA-CC
%   vehicles).

% \item \textbf{Communication device} (Used Types SA-ACC, SA-CC, and SA-Com vehicles).

% \commentp{This is a significant weakness of the paper.  "Whole-row
% request", "constant-velocity request", and "anchor request" haven't
% been defined yet.  They're used below, but really only "anchor
% request" is defined later in the paper (section VI). I think it makes
% more sense to merge section VI into section III and clearly define the
% types of requests that can be defined using constraints.  These
% requests could also be used by fully autonomous vehicles, so there's
% no reason that it needs to happen after section IV.  We can then
% emphasize that the traditional AIM system can also be implemented
% using these constraint-based requests (since a normal "fixed arrival,
% fixed trajectory" request is also possible).}

% us
% to design a new reservation systems that allows semi-autonomous
% vehicles to make reservations 



%%% Local Variables: 
%%% mode: latex
%%% TeX-master: "main"
%%% End:
