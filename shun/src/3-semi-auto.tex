\section{Semi-Autonomous Vehicles}
\label{sec:vehicles}

We use the term \emph{semi-autonomous vehicles} to
refer to vehicles with limited autonomous driving and wireless
communication capabilities.  While these vehicles are not fully
autonomous, they are able to follow a \emph{limited}
number of predictable trajectories at intersections more precisely
than human drivers.  This ability allows them to utilize our
constraint-based reservation system to make reservations in the same
manner as fully autonomous vehicles.

Our reservation system is general enough to accept
reservation requests from \emph{any} semi-autonomous vehicles that are
capable of following some trajectories and communicating with the IM\@.
To facilitate our discussion, we will focus on semi-autonomous
vehicles which use the following set of equipment that is readily
available.

\begin{small_ind_s_itemize}

\item \textbf{Communication device (Com)}:
a component in a vehicle's on-board electronic system that enables the
vehicle to wirelessly communicate with the transportation
infrastructure including the IM\@.  The communication is bidirectional:
the messages sent from the IM is presented to the human driver on
the LCD screen of an on-board navigation system or on a smartphone, and
the human driver makes decisions on the user interface of the device.
The device is also hooked up with the odometer, GPS, and other sensors
such that it can send these sensing information along with the request
messages to the IM.

\item \textbf{Simple Cruise control (CC)}:
An optional speed control subsystem in vehicles' drivetrain that
automatically controls the vehicle speed by taking over the throttle
of the vehicles.  With the help of cruise control systems, vehicles
can maintain a steady constant velocity more precisely than human
drivers can manually.

\item \textbf{Adaptive cruise control (ACC)}:
an advanced cruise control system that automatically adjusts the speed
of a vehicle in order to maintain a certain distance from vehicles
ahead. To achieve this car-following maneuver,
ACC uses on-board distance sensors coupled with cruise control
in a feedback loop.

\end{small_ind_s_itemize}

All of this equipment give semi-autonomous vehicles \emph{some} of
the functionality of autonomous vehicles, though human drivers still
retain some control of the vehicles.  We can equip a semi-autonomous
vehicle with more than one of these devices.  Next, we introduce three
types of semi-autonomous vehicles that we envision utilizing this
equipment.

\begin{small_ind_s_itemize}
\item \textbf{Type SA-ACC Vehicles}: Utilizing adaptive cruise
control to enter an intersection by either moving straight through the
intersection or following another vehicle.
\item \textbf{Type SA-CC Vehicles}: Using simple cruise control only
to enter an intersection at a constant velocity in a straight line.
\item \textbf{Type SA-Com Vehicles}: Reserve an entire lane in an
intersection such that the human driver can get through the
intersection without the help of any autonomous control device; thus
only the communication device is needed.
\end{small_ind_s_itemize}

%%% Local Variables: 
%%% mode: latex
%%% TeX-master: "main"
%%% End:
