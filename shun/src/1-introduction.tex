\section{Introduction}
\label{sec:introduction}

Recent robotic car competitions and demonstrations have convincingly
shown that autonomous vehicles are feasible with the current
generation of hardware~\cite{mybib:Darpa07Urban}. Looking ahead to the
time when autonomous cars will be common, Dresner and Stone proposed a
new intersection control protocol called \emph{Autonomous Intersection
Management} (AIM) and showed that by leveraging the control and
network capabilities of autonomous vehicles it is possible to design
an intersection control protocol that is much more efficient than
traffic signals~\cite{bib:Dresner08Multiagent}.  By removing human
factors from control loops, autonomous vehicles, with the help of
advanced sensing devices, can be safer and more reliable than human
drivers.  The AIM protocol exploits the fine control of autonomous
vehicles to allow more vehicles simultaneously to cross an
intersection, thus effectively reducing the delay of vehicles by
orders of magnitude compared to traffic
signals~\cite{bib:Fajardo12Automated}.

AIM is designed for the time when vehicles are autonomous.  We,
however, anticipate that there will be a long transition period during
which most vehicles have some but not all capabilities of fully
autonomous vehicles.  In fact, this transition period has already
begun. Since the late 1990s, adaptive cruise control systems and lane
departure warning systems have become widely available as optional
equipment on luxury production vehicles. While
AIM provides a significant efficiency improvement at intersections
when all cars are autonomous, the benefits are minimal even when as
few as 10\% of the vehicles are driven by humans (Figure 16
in~\cite{bib:Dresner08Multiagent}).  The requirement that most,
if not all,
vehicles are fully autonomous is a key obstacle to the adoption
of AIM-like intersection control when most vehicles are not fully autonomous.

%%% Local Variables: 
%%% mode: latex
%%% TeX-master: "main"
%%% End:
