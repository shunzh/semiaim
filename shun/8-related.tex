\section{Related Work}
\label{sec:related}

The main context of our work is an extension to the FCFS policy
proposed by Dresner and Stone~\cite{bib:Dresner08Multiagent}. Their
experimental results indicated that a mixture of human-driven vehicles
and autonomous vehicles is possible, and leads to better performance
than having all human-driven vehicles, which is the current status
quo.  However, their experiments indicated that the impact of
autonomous vehicles is expected to be relatively small until almost
all (90-95\%) of the vehicles on the road are
autonomous~\cite{bib:Dresner07Sharing}.  Our extension to embrace
semi-autonomous vehicles shows significant performance benefits all
along the technology penetration curve.

Our work is similar to the analysis of adaptive cruise control
performance by Jerath and Brennan, who showed that by introducing
adaptive cruise control vehicles into traffic, the vehicles would have
a more \textit{condensed} formation, thus increasing the efficiency of
traffic~\cite{bib:Jerath10adaptive}.  While the analysis of Jerath and
Brennan is focused on highways, we are focusing on the
intersection---a place more critical both from the points of view of
congestion and safety.

There is limited literature on understanding changes to road
infrastructure that can facilitate vehicular autonomy.  One such
project on jointly optimizing autonomous vehicles and road
infrastructure is the PATH program, which relies on magnetic markers
in the roadway for measuring steering angle and vehicle
movements~\cite{mybib:Shladover91Automated}.  Kolodko and Vlacic used
golf-cart-like Imara vehicles in evaluating an autonomous
intersection~\cite{Kolodko03:INRIA}.  Our work differs from these
previous research by incorporating semi-autonomous vehicles into the
mix, along with autonomous and human-driven vehicles.

Another way to manage autonomous vehicles at intersections is based on
Vehicle-to-Vehicle (V2V) communication, such that vehicles coordinate
in a peer-to-peer fashion when crossing the
intersection~\cite{my_naumann97:intersection,ATT08-vanmiddlesworth}.
Naumann {\em et al.} investigated a distributed policy that uses
virtual ``tokens'' that a vehicle must possess to cross certain
contested areas of the intersection~\cite{my_naumann97:intersection} and
formally evaluated it using petri-nets.  VanMiddlesworth {\em et al.}
developed an AIM-inspired protocol that enables vehicles to ``call
ahead'' to reserve space-time in the
intersection~\cite{ATT08-vanmiddlesworth}.  The anchor requests in
Section~\ref{sec:anchor} can be implemented using V2V communications.


% no
% centralized server is required 
% (i.e., there is no single point of
% failure) and 



% The vast majority of research on autonomous vehicles focuses on how to
% ensure they run on existing road infrastructure; 

% Other researchers have investigated autonomous intersections using
% real systems involving multiple mobile vehicles.  
% For example, 

% Their protocol outperformed
% the traditional stop sign in light traffic. 


% there are two key differences.  First, 
% Second, we study five types of vehicles,
% compared to the three types considered by Jerath and Brennan.

% Although we are not aware of any other work that is directly concerned
% with the interaction of semi-autonomous and autonomous vehicles,
% research on vehicular autonomy in general has made significant
% progress in recent years.  This was in part due to a series of robotic
% car competitions such as the \emph{DARPA Grand
% Challenges}~\cite{DARPAGrandChallenge}.  These competitions
% accelerated the development of autonomous vehicles to the point where
% the technical challenge of open-road autonomous driving is considered
% by some to be essentially solved~\cite{bib:Dresner08Multiagent}.  When
% pushed to extremes, autonomous vehicles can even outperform many
% human drivers in carrying out intricate
% maneuvers~\cite{Squatriglia2010}. The non-technical barrier for the
% adaptation of autonomous vehicles are largely traffic laws and
% regulations, though progress is also being made in these
% areas~\cite{calo2011-nevada}.


% The AIM protocol~\cite{bib:Dresner08Multiagent, mybib:Fajardo12Automated,
% mybib:Quinlan10Bringing} is a vehicle-to-infrastructure (V2I) mechanism in
% which vehicles request space-time in the intersection for their
% trajectories prior to arriving at the intersection; a server at the
% intersection handles these requests, granting or rejecting
% reservations using a grid-based collision detection scheme. This
% protocol is enhanced to reduce network traffic and increase safety
% using spatial-temporal buffers surrounding the
% vehicles~\cite{mybib:Fajardo12Automated}.

% We implemented a slightly
% simplified version of the protocol in~\cite{ATT08-vanmiddlesworth}
% that does not use estimated time of arrival and demonstrated that it
% also outperforms a stop sign in light traffic.

% Other researchers have investigated autonomous intersections using
% real systems involving multiple mobile vehicles.  For example, Kolodko
% and Vlacic used golf-cart-like Imara vehicles in evaluating an
% autonomous intersection~\cite{Kolodko03:INRIA}. 
% In their study, all
% vehicles must come to a complete stop at the intersection irrespective
% of traffic conditions.  Our work differs from this, and all other
% previous research by incorporating semi-autonomous vehicles into the
% mix, along with autonomous and human-driven cars.



% \section{Related Work}

% The main context of our work is an extension to the FCFS policy
% proposed by Dresner and Stone~\cite{bib:Dresner08Multiagent}. Their
% experimental results indicated that a mixture of human-driven vehicles
% and autonomous vehicles is possible, and leads to better performance
% than having all human-driven vehicles, which is the current status
% quo.  However, their experiments indicated that the impact of
% autonomous vehicles was relatively small until almost all (90-95\%) of
% the vehicles on the road are autonomous.  Our extension to embrace
% semi-autonomous vehicles, shows significant performance benefits all
% along the technology presentation curve.

% Our work is similar to the analysis of adaptive cruise control
% performance by Jerath and Brennan~\cite{bib:Jerath10adaptive}.  Jerath
% and Brennan showed that by introducing adaptive cruise control
% vehicles into traffic, the vehicles would have a more
% \textit{condense} performance, thus increasing the efficiency of
% traffic.  There are two key differences.  First, we are focusing on
% the intersection---a place more critical both from the points of view
% of congestion and safety, while the analysis of Jerath and Brennan is
% focused on highways.  Second, we study five types of vehicles,
% compared to the three types considered by Jerath and Brennan.  




%The
%features of these types of vehicles are various. We are able to see
%which features would increase the efficiency of the intersection the
%most.


% We did not require platooning in our analysis. Vehicles appear randomly
% at the spawning point of each lane. This is also the case for other
% policies that we are going to compare with in this paper.  Our
% analysis is also universal, not restricted to any manufacture. We will
% list the features of each type of semi-autonomous vehicles. Any type
% of cruise control vehicle or adaptive cruise control vehicle that
% satisfy these features can all be applied to the policy mentioned in
% our paper.


%%% Local Variables: 
%%% mode: latex
%%% TeX-master: "main"
%%% End:
