\section{Interacting with Intersection Managers}
\label{sec:interaction}

Due to limited capabilities semi-autonomous vehicles cannot
make the same kind of requests as fully-autonomous vehicles does.
And this greatly affects the interaction between semi-autonomous
vehicles and IMs.

The interaction between autonomous vehicles and semi-autonomous
vehicles are quite different.

% is different from the interaction between autonomous vehicles.
% The main difference lies in the requests made by semi-autonomous
% vehicles,

% To illustrate these features, we consider the types of reservation
% requests and possible outcomes for each type of vehicle.  Note that at
% any time, a human driver could intervene and simply follow the traffic
% signal.

\noindent
\textbf{Anchor Requests}
We also introduce a new type of reservation requests
called \emph{anchor requests}.  An anchor request is used for vehicles
whose entry conditions and trajectories depend on other vehicles.
For example, a vehicle $\veh$ with adaptive cruise control can make an
anchor request such as $Anchor(T, \vin, D)$, which means that
$\veh$ will follow another vehicle with the VIN number $\vin$ 
and maintain a following distance less than or equal to $D$.  The IM
considers the two vehicles as being a single, longer vehicle, with no
free space between them.

In their current form, anchor requests are designed specifically for
semi-autonomous vehicles with adaptive cruise control. However, our
formulation is general enough to include other kinds of cooperative maneuver
such as platooning~\cite{bib:Sheikholeslam90Longitudinal}.

\noindent
\textbf{Constant-Velocity Requests}
The constant velocity requests are very special.

\noindent
\textbf{Whole-Row Requests}
The vehicle with this feature can propose a \textit{whole-row
request}. This is a very strong request and is only likely to be
confirmed in very light traffic.  The IM considers whether, with
enough time and space reserved, it can safely go through the
intersection.


\subsection{Interaction Loops}

Since there are many different types of requests, we can combine them
in many different ways.  Here we describe the interactions for the
semi-autonomous vehicles we consider in this paper.


\textbf{Type B}

\begin{enumerate}

\item Propose an anchor request. If the vehicle in front of it is
semi-autonomous and is going in the same direction, then if they can
both get reservations, the request is confirmed. The vehicle can
follow and enter.

\item If request denied, propose a constant-velocity request. If
keeping the current velocity it can traverse the intersection, then
request confirmed.

\item If request denied, propose a whole-row request. If there is no
conflict if this vehicle enters directly, then request confirmed.

\item If denied, decelerate to stop, and retry step 3 or pass control
to the human.

\end{enumerate}

\textbf{Type C}

\begin{enumerate}

\item Propose a constant-velocity request. If it keeps the current
velocity and can enter the intersection, then request confirmed.

\item If request denied, propose a whole-row request. If 
keeping the current velocity it can traverse the intersection, then
request confirmed.

\item If denied, decelerate to stop, and retry step 2 or pass control
  to the human.

\end{enumerate}

\textbf{Type D}

\begin{enumerate}

\item Propose a whole-row request. If keeping the current velocity it
  can traversethe intersection, then request confirmed.

\item If denied, decelerate to stop, and retry step 1 or pass control
  to the human.

\end{enumerate}



% For adaptive cruise control, there is an anchor request (intention, distance)
% - before and after the followed vehicle having reservation
% - but even after the followed vehicle has a reservation, it may break
% the reservation.

% The conditions that has no need to reserve tiles
% before the anchor will break
% - the leader makes a reservation that does not fit the intention of
% the follower.  (the problem is that if it is delayed request, it
% requests cancellation request, which is not good.)
% Thus we need delayed confirmation
% - the leader makes a reservation that is out of the capability of
% the follower. (no reservation is made.)
% - the leader fails to follow the reservation. The vehicle is
% responsible for the detection and then do not enter the intersection.

% If the leader do not make reservation, the follower will not
% rejection.  But it will not have any confirmation either.

% In the case of chaining operation, the confirmation are made in order.



% \subsubsection{The interpretation of Requests}

% % \subsection{The Physical Meaning of Requests}

% Ultimately, the constraint requests the limitation of the
% trajectories of the vehicles. 

% What are the basic variables of a trajectory?r
% - arrival time
% - arrival velocity
% - trajectories 
% - acceleration schedule
% - physical characteristic of the vehicles

% The trajectories are in the form of 
% the boundary of geometric shapes.
% Ultimately, it becomes trajectory bundles. 
% (Maybe we don't deal with it).


% %%%%% Model

% The IM needs to generate the model of a given
% requests.

% What are the set of all possible models?

% Model existence?

% If no model exists, the IM will reject it.

% Let's skip this part first.


% \subsection{New Confirmation Messages}


% IM can return confirmation with a range of trajectories?
% The range determined by the vehicle capacity and the preferences,
% as well as the set of tiles reserved.


% \section{The IM algorithm}

% The objective of the IM is to solve the above constraint system.


% We formulate the IM as a planning problem.

% The objective of the IM algorithm is to generate an acceleration
% profile that satisfies the intention statement, while
% guarantees that there will be no collisions.


\subsection{Safety}

Safety is a common (and quite reasonable) concern when involving human
drivers in the control loop of autonomous vehicles.  In this section,
we describe how our proposed protocol can be realized safely, as long
as no driver runs a red light (without a reservation granted).

For the drivers of vehicle types B, C, and D, we assume inclusion in
the vehicle of a single button that signals the driver agent to ask
for a reservation.  We opt for such an active ``opt-in'' interface so
as to allow drivers to drive as they do today if they do not signal
their intention for a semi-autonomous reservation.  It is also
important that there is also a clear indicator (such as a green light)
installed in the car which indicates when the request has been
confirmed.  The driver must be clear about the current situation and
not surprised by the sudden autonomous actions of his vehicle.

\noindent
\textbf{Types B and C.} When the reservation is granted, the vehicle
starts controlling the speed.  If the person touches the brakes or
accelerator, the person regains control, and the reservation is
cancelled.

There is a point (near the entry of intersection) where it is no
longer possible to brake before entering the intersection. At this
point, the person loses the ability to take back control of the speed
until leaving the intersection (presumably with an emergency override
option for extreme situations).  This constraint is based on the
possibility that if the driver were to change its velocity, its needed
reservation would be modified.  The IM cannot guarantee that the
modified reservation would be able to be confirmed.

Drivers can steer in type C vehicles. For type B, the vehicle might
follow a vehicle in front of it to make a turn. The driver would lose
control over the steering wheel also in this case.

\noindent
\textbf{Type D.} The driver can safely enter the intersection when
the request is confirmed. If the request is confirmed and the driver
decides not to enter, it still does no harm (even though the tiles it
requested are wasted).

% \section{Definitions}

% A road $r$ consists a finite number of lanes $l_1$, $l_2$, \ldots,
% $l_n$.  We write $r = (1_1, l_2, \ldots l_n)$.
% All lanes in a road are in the same direction.
% An \emph{entry} road of an intersection is a road
% from which traffic flows into the intersection.
% An \emph{exit} road is a road from which traffic flows out of
% an intersection.


% \section{A General AIM architecture}

% We generalize the AIM architecture as follows.

% \commentc{Maybe draw a diagram to show to how AIM works}

% This framework can be considered as an extension to AIM.
% In AIM, each request is a $4$-tuple $\langle t, v, l_1, l_2 \rangle$.
% \commentc{Fit AIM requests into the next framework.}

% In AIM, these values are *exact*. However, in SemiAIM,
% these values can be constrained.

% No need to think about the acceleration schedule yet.

% In essence, no human intervention is involved in the control loop.


%%% Local Variables: 
%%% mode: latex
%%% TeX-master: "main"
%%% End:
